\begin{abstract}
O objetivo deste trabalho é o desenvolvimento de um \textit{Q\&A System}, isto é, um sistema de perguntas e respostas aplicando os conhecimentos adquiridos ao longo do semestre na unidade curricular de \textit{Gramáticas na Compreensão de Software} sobre gramáticas de atributos, gramáticas independentes de contexto, qualidade de gramáticas, entre outros. 

Na secção \ref{sec:areaCon} damos a conhecer e aprofundamos um pouco sobre a área de conhecimento escolhida. Na secção \ref{sec:funcs}, são indicadas as funcionalidades suportadas pelo sistema desenvolvido. Numa outra secção, \ref{sec:gic}, é explicada a GIC construída e a estrutura lógica do programa.Na secção \ref{sec:ga}, são apresentadas as funções e uso de atributos que suportam as funcionalidades disponibilizadas. Por fim, na secção \ref{sec:conc}, é feita uma análise crítica ao trabalho e possíveis melhorias.

\end{abstract}

% Background: Latest information on the topic; key phrases that pique interest (e.g., “…the role of this enzyme has never been clearly understood”).
% Objective: Your goals; what the study examined and why.
% Methods: Brief description of the study (e.g., retrospective study).
% Results: Findings and observations.
% Conclusions: Were these results expected? Whether more research is needed or not?