\qquad A área de conhecimento escolhida para o trabalho desenvolvido foi o \textit{FAQ (Frequently Asked Questions)}~\cite{FAQ} da Universidade do Minho. O seu objetivo é responder às questões mais frequentes colocadas pelos alunos diminuindo o volume de emails enviados e de idas aos vários serviços da universidade ( e.g. Serviços Académicos, Serviços de Ação Social, Serviços de Documentação, Serviços de Relações Internacionais, Direção de Recursos Humanos, ... ). Essencialmente aumentar a sua eficiência. Assim, tem como vantagem a redução do tempo de espera quer de resposta aos \textit{emails} quer das filas no atendimento presencial.

 O tema da informação guardada na base de conhecimento é a Universidade do Minho, nomeadamente, informação relativa ao calendário escolar, pagamento de propinas, admissão aos cursos, recrutamento de investigadores, residências universitárias, ECTSs, inscrições, campi e matrículas, entre outros. Quanto aos utilizadores, o sistema destina-se a uso por parte de alunos (que já frequentem a universidade ou que o pretendam fazer no futuro), \textit{alumni}, corpo docente, investigadores, funcionários ou membros da direção.

Alguns exemplos de perguntas possíveis a que o \textit{Q\&A System} poderá dar resposta (contando que a informação seja corretamente inserida na base de conhecimento) são:
\begin{itemize}
    \item O que são ECTSs?
    \item Onde me posso candidatar à Universidade do Minho?
    \item Como posso fazer \textit{Erasmus}?
    \item Onde me devo dirigir para assuntos relacionados com bolsas de estudo?
    \item Qual é o calendário escolar da Escola de Engenharia?
\end{itemize}