\qquad De forma a pôr o trabalho funcional, isto é, criar os mecanismos de resposta procedeu-se à transformação da gramática de uma GIC (gramática independente de contexto) para uma GA (gramática de atributos).

A criação da base de conhecimento é feita através da produção '\textit{bcQAS}'. Os pares pergunta-resposta são lidos do ficheiro de teste e guardados num \textit{hashmap} cujas chaves são os 'tipos' das intenções (e.g. 'Como','Quando', ...) e os valores são Arraylists de pares (intencao, resposta). Um objeto do tipo \texttt{Par} contém o tipo, a ação, as keywords e informação associada, correspondendo essencialmente a um par pergunta-resposta da base de conhecimento. 
Esta decisão foi tomada para tornar a procura mais eficiente.

\begin{verbatim}

bcQAS returns [HashMap<String, ArrayList<Par>> bc]
@init{$bcQAS.bc = new HashMap<String,ArrayList<Par>>();}
    : t1=par [$bcQAS.bc] (t2=par [$t1.bcOut] {$t1.bcOut = $t2.bcOut;})*
    ;

par [HashMap<String, ArrayList<Par>> bcIn] 
    returns [HashMap<String, ArrayList<Par>> bcOut]
    : '(' intencao ';' resposta')' 
    {
        $intencao.p.resposta = $resposta.val;
        ArrayList<Par> aux = $bcIn.get($intencao.p.tipo);
        if(aux==null) aux = new ArrayList<Par>();
        aux.add($intencao.p);
        $bcIn.put($intencao.p.tipo,aux);
        $bcOut = $bcIn;
    }
    ;

intencao returns [Par p]
        : tipo ',' acao ',' keywords
        {
            $intencao.p = new Par();
            $intencao.p.tipo = $tipo.val;
            $intencao.p.acoes = $acao.list;
            $intencao.p.keywords = $keywords.list;
        }
        ;

\end{verbatim}

Por outro lado, é importante também nos debruçarmos sobre o processo de resposta.

A produção responsável por responder às questões colocadas é a '\textit{questao}'. Aqui são reconhecidas as várias palavras que compõem uma pergunta que podem já existir na base de conhecimento, nomeadamente, tipos, ações e keywords, mas também palavras que não sejam conhecidas. Cada questão é composta por pelo menos uma palavra e terminada por pelo menos um ponto terminal.

\begin{verbatim}

questao [HashMap<String, ArrayList<Par>> bc]
@init{
    ArrayList<String> tipos = new ArrayList<String>();
    ArrayList<String> acoes = new ArrayList<String>();
    ArrayList<String> keywords = new ArrayList<String>();
    ArrayList<String> palavras = new ArrayList<String>();
    StringBuffer question = new StringBuffer();
}
    :( PALAVRA { palavras.add($PALAVRA.text);
                 question.append($PALAVRA.text).append(" ");
               } 
    | tipo { tipos.add($tipo.val); 
             question.append($tipo.val).append(" ");
           } 
    | acao { acoes.add($acao.val); 
             question.append($acao.val).append(" ");
           } 
    | keyword { keywords.add($keyword.val); 
                question.append($keyword.val).append(" ");
              } 
     )+ (PONTOTERMINAL {question.append($PONTOTERMINAL.text);} )+
     
    {getAnswer($bc,question,tipos,acoes,keywords,palavras);}
    ;
\end{verbatim}

Quando é reconhecida uma pergunta completa, procede-se à tentativa de obtenção de resposta. Esta fica ao encargo do método \texttt{getAnswer} definido no \texttt{@members}. Esta tenta obter um match completo das palavras extraídas da questão, com a parte da intenção dos pares (intencao, resposta) que contêm o conhecimento. Se isto não for possível, tenta otimizar o mais possível.

Exemplificando, para a base de conhecimento abaixo os resultados esperados seriam os seguintes.

\begin{verbatim}
BC:
( Quando , pagar , [ propinas ] ; ' Até dia 23 de julho .' )
( O que , ser , [ ECTS ] ; ' European Credit Transfer System .' )
( Qual , ser , [ email , diretor ,  curso ] ; ' ddd@ddd.com. ' )
( Como , aceder , [ calendário , escolar ] ; 
       ' O calendário escolar está disponível no portal académico. ' )
( Como , imprimir , [ calendário , escolar ] ; 
       ' O calendário escolar pode ser imprimido através do portal académico. ' )
( Onde , inscrever , [ exame , época , especial ] ; 
       ' Incrições para exame de Época Especial são feitas no portal académico. ' )
( Onde , inscrever , [ exame , época , recurso ] ; 
       ' Não é necessária incrição para realizar exame de recurso. ' )

\end{verbatim}

\begin{verbatim}

Gostava de saber quando se pagam as propinas de 2018/19 ...
R0:' Até dia 23 de julho .'

Siri, diz-me: O QUE SÃO ECTS ?!
R0:' European Credit Transfer System .'

Qual é o nome do reitor ?
Não foi encontrada resposta à sua pergunta.

ola, onde nos devemos inscrever para o exame de Época Especial ?
R0:' Incrições para exame de Época Especial são feitas no portal académico. '

Onde nos devemos inscrever para o exame de Época de Recurso ?
R0:' Não é necessária incrição para realizar exame de recurso. '

Onde nos devemos inscrever para o exame de recurso ?
R0:' Não é necessária incrição para realizar exame de recurso. '

Onde nos devemos inscrever para o exame ?
R0:' Incrições para exame de Época Especial são feitas no portal académico. '
R1:' Não é necessária incrição para realizar exame de recurso. '

Como aceder para imprimir o calendário escolar ?
R0:' O calendário escolar está disponível no portal académico. '
R1:' O calendário escolar pode ser imprimido através do portal académico. '
\end{verbatim}



A Gramática de Atributos e versão final do \textit{Q\&A System} com área de conhecimento sendo a FAQ da Universidade do Minho pode ser visualizada em anexo na secção \ref{GA}.
 